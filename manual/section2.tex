\section{Installation}

So you decided to give Jhin3 a try? Great!
To run Jhin3, you need to have Java (version 11 or higher) installed. This makes Jhin3 compatible with all the major operating systems such as Linux, macOS and Windows. Jhin3 itself is nothing more than a jar file. This means that you can either install Jhin3 to a system directory or just put the jar file somewhere you like in your user directory. Decide for yourself what suits to your needs.

\subsection{Getting the jar file}

You can either download a pre-built version of Jhin3 from the GitHub releases page or you can build Jhin3 yourself.

\subsubsection{Downloading a pre-built version (recommended)}

Go to the GitHub releases page of Jhin3 (\href{https://github.com/hannesbraun/jhin3/releases}{or click here}). Select the latest version and download the associated zip file. It is called something like ``jhin3-2020.0.0.zip''.
Unpack the zip file and you will find the jar file inside.

\subsubsection{Building from source}

To build Jhin3 from source, make sure you have a JDK (version 11 or higher) installed as well as Maven.

Clone the Jhin3 repository to your computer and navigate to its root directory. Now, just run this command:
\begin{lstlisting}[language=bash]
mvn package
\end{lstlisting}

Inside of the target directory, you should find a file called ``jhin3-\{version\}-jar-with-dependencies.jar''.

\subsection{Optional: installing Jhin3}

Right now, you need to type
\begin{lstlisting}[language=bash]
java -jar /path/to/jhin3.jar
\end{lstlisting}
every time you want to run Jhin3. To simplify this, you can install Jhin3 to a system directory and execute it just by typing
\begin{lstlisting}[language=bash]
jhin3
\end{lstlisting}

\subsubsection{Linux/macOS}

In the /opt directory of your system, create a new directory called ``jhin3''. Rename the Jhin3 jar file to ``jhin3.jar'' and move it into this new directory.
Next, create a file called ``jhin3'' inside of ``/usr/local/bin''. This serves as a wrapper for the actual jar file.
Copy the following code inside that file:

\begin{lstlisting}[language=bash]
#!/bin/sh

java -jar /opt/jhin3/jhin3.jar "$@"
\end{lstlisting}

Make sure this file is executable. You should now be able to run Jhin3 anywhere by typing:
\begin{lstlisting}[language=bash]
jhin3
\end{lstlisting}

To update Jhin3, replace ``/opt/jhin3/jhin3.jar'' with the new jar file.
