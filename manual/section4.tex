\section{Using Jhin3}

\textbf{Important note:} For windows, use javaw instead of java.

\subsection{Command line options}

The following command line options are available:
\begin{itemize}
\item ``-b [buffer size]'': audio buffer size in ms
\item ``-c [path]'': path to the configuration file
\item ``-t [theme]'': theme to use (currently available: ``2019'', ``2020'', ``bsod'' and ``frost-archer'')
\item ``--version'': only prints the header including the version number (and exits afterwards)
\end{itemize}

\subsection{General usage}

To switch between the soundboard and the time window press ``ctrl-F''.

In the soundboard window, you can press any of the mapped keys (defined inside of the given configuration file). This will trigger the sound depending on its type.
The column ``state'' shows if the sound is fading in, playing or fading out. The progress column shows the playing progress of the last started instance of the sound.

In the time window, you can start and stop the timer by pressing ``S''. To reset the timer, press ``C''.
Changing the timer duration is possible by pressing ``X''. The maximum length is 99 hours, 59 minutes and 59 seconds. The input has to be of the form HH:mm:ss.

The four stopwatches can be started and stopped by pressing ``1'', ``2'', ``3'' or ``4''. To reset them, press ``Q'', ``W'', ``E'' and ``R'' respectively.

To exit Jhin3 press the esc-key.